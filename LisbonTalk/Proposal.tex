\documentclass{article}
\usepackage{amsmath}
\usepackage{caption}
\usepackage{placeins}
\usepackage{graphicx}
\usepackage{subcaption}
\usepackage{setspace}
%\usepackage[active,tightpage]{preview}
\usepackage{natbib}
\bibpunct{(}{)}{,}{a}{}{;} 
\usepackage{url}
\usepackage{nth}
\usepackage{authblk}
\usepackage[brazilian]{babel}
\usepackage[utf8]{inputenc}
\usepackage[T1]{fontenc}
% for the d in integrals
\newcommand{\dd}{\; \mathrm{d}}
\newcommand{\tc}{\quad\quad\text{,}}
\newcommand{\tp}{\quad\quad\text{.}}
\defcitealias{HMD}{HMD}
\usepackage{setspace}
\newcommand\ackn[1]{%
  \begingroup
  \renewcommand\thefootnote{}\footnote{#1}%
  \addtocounter{footnote}{-1}%
  \endgroup
}
\begin{document}

\title{Demographic time, metabolism, and structure: Comments on Portugal and the EU\\ \vspace{1em}
\small{Presentation and workshop to be held at the} \\ 
\small{V congreso Portugu\^{e}s de demografia} \\
\small{6-7 October, 2016}}
\author[1]{Tim Riffe\thanks{riffe@demogr.mpg.de} \\ Max Planck Institute for Demographic Research}


\maketitle
\onehalfspacing
\section*{Description}
We'll use different aspects of demographic structure (both temporal structure and other elements) to characterize the demographic state and demographic potential of Portugal. This may include implementing non-standard notions of population renewal, population universe, and population mass. I will introduce some general concepts, suggest how to interpret and implement them, then we'll work with data in either spreadsheets or \texttt{R} (as you like) to come to some broad conclusions that will either refute or support the title of this conference. Our exploration will consist in a few fun exercises and visualizations that you'll easily be able to extend. Data may come from a variety of publicly available sources, such as HMD, HFD, and others, and it will be provided in the workshop.


%\bibliographystyle{plainnat}
%\bibliography{C:/Users/remunda/Documents/Travail/Masterbib}
%\bibliography{Masterbib}

\end{document}